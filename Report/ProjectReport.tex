\documentclass{article}

% if you need to pass options to natbib, use, e.g.:
% \PassOptionsToPackage{numbers, compress}{natbib}
% before loading nips_2016
%
% to avoid loading the natbib package, add option nonatbib:
% \usepackage[nonatbib]{nips_2016}

\usepackage[final]{nips_2016}

% to compile a camera-ready version, add the [final] option, e.g.:
% \usepackage[final]{nips_2016}

\usepackage[utf8]{inputenc} % allow utf-8 input
\usepackage[T1]{fontenc}    % use 8-bit T1 fonts
\usepackage{hyperref}       % hyperlinks
\usepackage{url}            % simple URL typesetting
\usepackage{booktabs}       % professional-quality tables
\usepackage{amsfonts}       % blackboard math symbols
\usepackage{nicefrac}       % compact symbols for 1/2, etc.
\usepackage{microtype}      % microtypography
\usepackage{amsmath}
\usepackage{graphicx}
\title{Parallel Data Mining}
\author{
	Jason Ma \& Kanishk Tantia\\
	Introduction to High Performance and Parallel Computing \\
	Harvey Mudd College \\
	\texttt{jyma@g.hmc.edu \& ktantia@g.hmc.edu}   
	}
% The \author macro works with any number of authors. There are two
% commands used to separate the names and addresses of multiple
% authors: \And and \AND.
%
% Using \And between authors leaves it to LaTeX to determine where to
% break the lines. Using \AND forces a line break at that point. So,
% if LaTeX puts 3 of 4 authors names on the first line, and the last
% on the second line, try using \AND instead of \And before the third
% author name.



\begin{document}
	\maketitle
	\begin{abstract}

	\end{abstract}
	
	\section{Introduction}	
		
	Data storage needs have grown exponentially as more data is increasingly digitized. IBM estimates that 2.5 Quintillion bytes of data are created daily. IDC research estimates that digital data growth will grow at a compound growth rate of 11.7\% through 2020. As of 2011, the demand for more storage has outpaced the growth of digital storage means, and by 2020, it is estimated that the demand will outpace the growth of digital storage by over 15,000 Exabytes of data per year. 
	
	Data Deduplication is an efficient manner in which data usage can be reduced, and has slowly gained traction over the last decade. Assuming that a large amount of the data being stored is redundant or "junk" data, data storage needs can be met by simply never storing multiple copies of data and instead just providing the same set of bytes when the data is required. 
	
	

	\section{Data Processing}

	
	\section{Algorithms Tested}

	\subsection{FastCDC}
	
	\subsection{RabinCDC}
	
	\subsection{AECDC}	
	
	\section{Experimental Approach}
	\subsection{Practical Traces}
	
	\subsection{Synthetic Traces}
	
	\section{Results}

	\section{Conclusions}

	\section{References}

	
\end{document}